% Options for packages loaded elsewhere
\PassOptionsToPackage{unicode}{hyperref}
\PassOptionsToPackage{hyphens}{url}
%
\documentclass[
]{article}
\usepackage{amsmath,amssymb}
\usepackage{iftex}
\ifPDFTeX
  \usepackage[T1]{fontenc}
  \usepackage[utf8]{inputenc}
  \usepackage{textcomp} % provide euro and other symbols
\else % if luatex or xetex
  \usepackage{unicode-math} % this also loads fontspec
  \defaultfontfeatures{Scale=MatchLowercase}
  \defaultfontfeatures[\rmfamily]{Ligatures=TeX,Scale=1}
\fi
\usepackage{lmodern}
\ifPDFTeX\else
  % xetex/luatex font selection
\fi
% Use upquote if available, for straight quotes in verbatim environments
\IfFileExists{upquote.sty}{\usepackage{upquote}}{}
\IfFileExists{microtype.sty}{% use microtype if available
  \usepackage[]{microtype}
  \UseMicrotypeSet[protrusion]{basicmath} % disable protrusion for tt fonts
}{}
\makeatletter
\@ifundefined{KOMAClassName}{% if non-KOMA class
  \IfFileExists{parskip.sty}{%
    \usepackage{parskip}
  }{% else
    \setlength{\parindent}{0pt}
    \setlength{\parskip}{6pt plus 2pt minus 1pt}}
}{% if KOMA class
  \KOMAoptions{parskip=half}}
\makeatother
\usepackage{xcolor}
\usepackage[margin=1in]{geometry}
\usepackage{graphicx}
\makeatletter
\def\maxwidth{\ifdim\Gin@nat@width>\linewidth\linewidth\else\Gin@nat@width\fi}
\def\maxheight{\ifdim\Gin@nat@height>\textheight\textheight\else\Gin@nat@height\fi}
\makeatother
% Scale images if necessary, so that they will not overflow the page
% margins by default, and it is still possible to overwrite the defaults
% using explicit options in \includegraphics[width, height, ...]{}
\setkeys{Gin}{width=\maxwidth,height=\maxheight,keepaspectratio}
% Set default figure placement to htbp
\makeatletter
\def\fps@figure{htbp}
\makeatother
\setlength{\emergencystretch}{3em} % prevent overfull lines
\providecommand{\tightlist}{%
  \setlength{\itemsep}{0pt}\setlength{\parskip}{0pt}}
\setcounter{secnumdepth}{-\maxdimen} % remove section numbering
% definitions for citeproc citations
\NewDocumentCommand\citeproctext{}{}
\NewDocumentCommand\citeproc{mm}{%
  \begingroup\def\citeproctext{#2}\cite{#1}\endgroup}
\makeatletter
 % allow citations to break across lines
 \let\@cite@ofmt\@firstofone
 % avoid brackets around text for \cite:
 \def\@biblabel#1{}
 \def\@cite#1#2{{#1\if@tempswa , #2\fi}}
\makeatother
\newlength{\cslhangindent}
\setlength{\cslhangindent}{1.5em}
\newlength{\csllabelwidth}
\setlength{\csllabelwidth}{3em}
\newenvironment{CSLReferences}[2] % #1 hanging-indent, #2 entry-spacing
 {\begin{list}{}{%
  \setlength{\itemindent}{0pt}
  \setlength{\leftmargin}{0pt}
  \setlength{\parsep}{0pt}
  % turn on hanging indent if param 1 is 1
  \ifodd #1
   \setlength{\leftmargin}{\cslhangindent}
   \setlength{\itemindent}{-1\cslhangindent}
  \fi
  % set entry spacing
  \setlength{\itemsep}{#2\baselineskip}}}
 {\end{list}}
\usepackage{calc}
\newcommand{\CSLBlock}[1]{\hfill\break\parbox[t]{\linewidth}{\strut\ignorespaces#1\strut}}
\newcommand{\CSLLeftMargin}[1]{\parbox[t]{\csllabelwidth}{\strut#1\strut}}
\newcommand{\CSLRightInline}[1]{\parbox[t]{\linewidth - \csllabelwidth}{\strut#1\strut}}
\newcommand{\CSLIndent}[1]{\hspace{\cslhangindent}#1}
\ifLuaTeX
  \usepackage{selnolig}  % disable illegal ligatures
\fi
\usepackage{bookmark}
\IfFileExists{xurl.sty}{\usepackage{xurl}}{} % add URL line breaks if available
\urlstyle{same}
\hypersetup{
  pdftitle={Thesis},
  pdfauthor={Anastasia Stuart},
  hidelinks,
  pdfcreator={LaTeX via pandoc}}

\title{Thesis}
\author{Anastasia Stuart}
\date{2024-05-29}

\begin{document}
\maketitle

\vspace*{\fill}

\noindent \textit{
I, Anastasia Stuart confirm that the work presented in this thesis is my own. Where information has been derived from other sources, I confirm that this has been indicated in the thesis.
} \vspace*{\fill} \pagenumbering{gobble} \newpage

\section*{Acknowledgements}\label{acknowledgements}
\addcontentsline{toc}{section}{Acknowledgements}

\newpage

\section*{Abstract}\label{abstract}
\addcontentsline{toc}{section}{Abstract}

\begin{verbatim}
Approx 300 words
\end{verbatim}

Lorem ipsum dolor sit amet, consectetur adipiscing elit. Nam et turpis
gravida, lacinia ante sit amet, sollicitudin erat. Aliquam efficitur
vehicula leo sed condimentum. Phasellus lobortis eros vitae rutrum
egestas. Vestibulum ante ipsum primis in faucibus orci luctus et
ultrices posuere cubilia Curae; Donec at urna imperdiet, vulputate orci
eu, sollicitudin leo. Donec nec dui sagittis, malesuada erat eget,
vulputate tellus. Nam ullamcorper efficitur iaculis. Mauris eu vehicula
nibh. In lectus turpis, tempor at felis a, egestas fermentum massa.

\newpage 
\pagenumbering{roman}
\setcounter{page}{1}

\pagenumbering{gobble}

\tableofcontents

\newpage

\pagenumbering{gobble}

\tableofcontents

\newpage

\listoffigures

\newpage

\listoftables

\newpage

\section*{Abbreviations}\label{abbreviations}
\addcontentsline{toc}{section}{Abbreviations}

\begin{tabbing}
\textbf{API}~~~~~~~~~~~~ \= \textbf{A}pplication \textbf{P}rogramming \textbf{I}nterface \\  
\textbf{JSON} \> \textbf{J}ava\textbf{S}cript \textbf{O}bject \textbf{N}otation \\  
\end{tabbing}

\newpage

\setcounter{page}{1}
\pagenumbering{arabic}
\doublespacing
\setlength{\parindent}{0.5in}

\section{Introduction}\label{introduction}

\subsection{Problem statement}\label{problem-statement}

Insomnia is the most common sleep complaint in Australia {[}cite{]},
affecting over \textbf{how many people} and leading to \textbf{health
burden} (Sweetman et al., 2021). It is associated with clinical
complaints of shortened overnight sleep, difficulty with sleep
initiation, and frequent overnight arousals causing clinically
significant distress or dysfunction in daily life (Association, 2022a).
However, there is a clinical population that experience non-restorative
sleep (NRS), characterised by poor subjective sleep quality despite
objectively normal polysomnographic recordings, leading to impaired
daytime functioning (Roth et al., 2010). This population is poorly
understood and not clinically managed \textbf{explain}.

The inability to identify objective markers of this \textbf{disease}
through PSG may result from the inability of current methods to
adequately measure sleep.

\textbf{People with NRS and Insomnia may experience subjective and
objective sleepiness differently to healthy controls, which impacts
their daily life and functioning. This could affect how they need to be
managed clinically, and greater understanding of this problem could lead
to improved outcomes.}

\begin{verbatim}
  ```what is the issue. impact. theoretical model. what is the solution. is there a difference in how NRS and ID percieve daytime tiredness, and is that related to delta power in previous nights sleep?
\end{verbatim}

\subsection{Background}\label{background}

\begin{itemize}
\tightlist
\item
  Definition and prevalence of insomnia

  \begin{itemize}
  \item
    (Sweetman et al., 2021)
  \item
    (Reynolds et al., 2019)
  \item
    Overview of symptoms
  \item
    Models of insomnia

    \begin{itemize}
    \tightlist
    \item
      stimulus control model
    \item
      3P model
    \end{itemize}
  \item
    Insomnia is associated with sleep homeostasis dysfunction, possibly
    due to decreased homeostatic drive or arousal interfering with the
    homeostatic process or its dissipation over the night.
    (Lunsford-Avery et al., 2021)
  \item
    lower sleep duration and lower NREM delta power than HC (Kao et al.,
    2021)
  \end{itemize}
\item
  Definition and overview of NRS
\end{itemize}

Non-restorative sleep is a population experiencing objectively normal
sleep as measured by PSG, however a feeling of being unrefreshed upon
awakening (Stone et al., 2008).

\begin{verbatim}
- lower nrem delta power than HC but same objective sleep time [@kao2021]

- [@stone2008]
  - variation in definition leads to impairment of research

- Definition and overview of sleep-state misperception
  - What are the neural mechanisms?
  - Why is it important? 
  - How does it affect people
  
  **interweave this in NRS, not actually measuring it in the study so perhaps not necessary to describe?**
  **could be mismeasurement, something to do with how we measure sleep and how we measure sleep is overestimating what we know**
  
  **what are the neural correlates? Is it deficits in delta waves? 


  Why are we not able to measure sleep-state misperception? Is it an issue in how we measure sleep?
\end{verbatim}

\subsection{Self-reported sleepiness}\label{self-reported-sleepiness}

\begin{itemize}
\tightlist
\item
  subjective sleepiness is a measure of an individual's self-assessed
  propensity to fall asleep in a particular situation. Persisistent
  sleepiness is debilitating and often associated with physical or
  emotional distress
\end{itemize}

``physiological indices of sleepiness did not occur reliably until
subjective perceptions fell between ``sleepy'' and ``extremely
sleepy-fighting sleep''; i.e.~physiological changes due to sleepiness
are not likely to occur until extreme sleepiness is encountered. ''
(Åkerstedt et al., 2009) - weak association (\[r\]=0.18) between
subjective fatigue and sleepiness in individuals with sleep disorders.
Analysis of variance testing showed strong association between those
patients with prominent fatigue and depressive symptoms (P \textless{}
0.01) and illness intrusiveness (P \textless{} 0.001). The findings
support the notion that subjective fatigue and sleepiness can be
independent manifestations of sleep disorders (Hossain et al., 2005) -
excessive sleepiness is regarded as one of the cardinal manifestations
of sleep disorders and often is accompanied by fatigue, many patients
with fatigue complain of insomnia and do not report falling asleep or
feeling sleepy at inappropriate times (Hossain et al., 2005)

\subsection{EEG drowsiness}\label{eeg-drowsiness}

Drowsiness is the experience of moving from wake to sleep, measured most
reliably through EEG.

\begin{itemize}
\tightlist
\item
  relevance to diagnosis and treatment of sleep disorders
\end{itemize}

current understanding:

\begin{itemize}
\item
  alpha waves: This brain activity can be easily identified by its
  topographic distribution (maximum amplitude over occipital regions),
  frequency range (8--13 Hz), and reactivity (it suffers a dramatic
  amplitude attenuation with the opening of the eyes). Drowsiness-alpha
  activity is typically characterized by decreased amplitude over
  occipital areas, as compared with the wakefulness-alpha rhythm,
  simultaneous to the appearance of a slower alpha pattern localized
  over anterior cortical regions. (Cantero et al., 2002)
\item
  detecting fatigue: algorithm (i) \(\frac {\theta + \alpha}{\beta}\),
  algorithm (ii) \(\frac {\alpha}{\beta}\), algorithm (iii)
  \(\frac {\theta + \alpha} {\alpha + \beta}\), and algorithm (iv)
  \(\frac {\theta}{\beta}\), were also assessed as possible indicators
  for fatigue detection. Results showed stable delta and theta
  activities over time, a slight decrease of alpha activity, and a
  significant decrease of beta activity (p \textless{} 0.05). All four
  algorithms showed an increase in the ratio of slow wave to fast wave
  EEG activities over time. Algorithm (i) (θ + α)/β showed a larger
  increase. (Jap et al., 2009)
\end{itemize}

\subsubsection{Explain brain waves and how they relate to
drowsiness}\label{explain-brain-waves-and-how-they-relate-to-drowsiness}

\begin{verbatim}
- AAC
- Slowing ratio
\end{verbatim}

\subsection{Power spectra}\label{power-spectra}

\begin{itemize}
\tightlist
\item
  The most common quantitative method employed in sleep studies is
  spectral analysis, which decomposes a time series of EEG data into
  power (squared amplitude) in frequency bins (mV2/bin) , can be
  expressed as absolute or relative to the summed power in all bins,
  spectral analysis may represent an objective method for examining the
  pathophysiological mechanisms underlying insomnia (Zhao et al., 2021)
\item
  Raw PSD has a straightforward connection to signal amplitude, with
  channels expressoinf larger signal amplitudes typically showing larger
  power, useful when absolute differences in signal amplitude are deemed
  meaningful (topographical analysis) (Cox \& Fell, 2020)
\end{itemize}

\subsection{Slow wave sleep}\label{slow-wave-sleep}

\begin{itemize}
\item
  EEG slow waves of NREM sleep occur when neurons become bistable and
  oscillate between two states: a hyperpolarized down-state
  characterized by neuronal silence (off-period), and a depolarized
  up-state during which neurons fire (on-period) (Steriade et al.,
  2001). During up-state, neurons fire at high frequencies typical of
  waking, and during down state there is a tonic cessation of firing
  activity in all cortical layers (Steriade et al., 1993)
\item
  slow oscillation is a travelling wave that originates at a definite
  site and travels over the scalp at an estimated speed of 1.2-7.0
  m/sec, waves originate more frequently in prefrontal-orbitofrontal
  regions and propagate in an anteroposterior direction (Massimini et
  al., 2004).
\item
  We identified two clusters of delta waves with distinctive properties:
  (1) a frontal-central cluster characterized by ∼2.5--3.0 Hz,
  relatively large, notched delta waves (so-called ``sawtooth waves'')
  that tended to occur in bursts, were associated with increased gamma
  activity and rapid eye movements (EMs), and upon source modeling
  displayed an occipital-temporal and a frontal-central component and
  (2) a medial-occipital cluster characterized by more isolated, slower
  (\textless2 Hz), and smaller waves that were not associated with rapid
  EMs, displayed a negative correlation with gamma activity, and were
  also found in NREM sleep. Therefore, delta waves are an integral part
  of REM sleep in humans and the two identified subtypes (sawtooth and
  medial-occipital slow waves) may reflect distinct generation
  mechanisms and functional roles. (Bernardi et al., 2019)
\item
  Insufficiency of slow-wave sleep may predict cognitive impairment and
  severity of chronic insomnia (Li et al., 2016)
\end{itemize}

\subsection{Delta power}\label{delta-power}

Importance of delta waves in sleep architecture - characteristics and
function - association with restorative sleep processes

NRS/ID/HC : quantitative differences in delta power - best established
method of sleep homeostasis - delta power dissipation overnight
(Lunsford-Avery et al., 2021) - insomnia patients exhibit a slower rate
in overnight delta decline compared to HC, not explained by differences
in total sleep time or wake after sleep onset. (Lunsford-Avery et al.,
2021)

correlation with daytime drowsiness:

\subsection{high frequency arousal}\label{high-frequency-arousal}

\begin{itemize}
\tightlist
\item
  (Zhao et al., 2021) meta-analysis found throughout wakefulness and
  sleep, patients with ID exhibited increased beta band power, although
  such increases sometimes extended into neighboring frequency bands,
  increased theta and gamma power during wake, increased alpha and sigma
  power during REM, decreased delta and increased theta, alpha, sigma
  power during NREM sleep.
\item
  ID is associated with significantly increased EEG activity in
  high-frequency bands (beta/gamma) during g reststate wakefulness,
  sleep-onset, non-rapid eye movement, may reflect cortical hyperarousal
  (Zhao et al., 2021)
\item
  no significant differences in waking or NREM sleep power were observed
  aross all frequency bands in PI (Wu et al., 2013)
\end{itemize}

\subsubsection{Differences of brain waves of people with insomnia/NRS
and healthy
controls}\label{differences-of-brain-waves-of-people-with-insomnianrs-and-healthy-controls}

\begin{verbatim}
- Increased slowing ratio
- Higher delta and theta power
    - Associated with increased sleepiness and cognitive implications
- 
\end{verbatim}

\subsection{approach}\label{approach}

\begin{itemize}
\tightlist
\item
  why are we doing things in the way we are doing? Integrate with theory

  \begin{itemize}
  \tightlist
  \item
    Link psychological construct to apporach you are using it to measure
    it
  \item
    Operationalise how you are going to measure constructs
  \end{itemize}
\item
  description of overall research study

  \begin{itemize}
  \tightlist
  \item
    what type of study it is,

    \begin{itemize}
    \tightlist
    \item
      observational, age and sex matched
    \end{itemize}
  \end{itemize}
\end{itemize}

\subsection{aim}\label{aim}

In order to examine regional differences in brain activity across three
populations, we used HD-EEG to measure \textbf{spectral power} during
resting wake and sleep. We aimed to see if there was a difference in the
correlation of objective sleepiness scores to objective measures of
drowsiness and if that was associated with topographical cluster
differences. Finally, we examined if the regional differences was
associated with delta power of previous nights sleep.

\subsection{Hypotheses}\label{hypotheses}

\begin{enumerate}
\def\labelenumi{\arabic{enumi}.}
\tightlist
\item
  KSS scores will be higher in the ID and NRS groups compared to healthy
  controls, indicating increased subjective sleepiness.
\item
  The correlation between KSS score and Slowing Ratio (SR) will differ
  significantly between the three groups. Healthy controls will have the
  strongest relationship between KSS score and SR, while NRS will have
  the weakest relationship.
\item
  Topography of channel-by-channel comparisons for normalised power
  spectral density will reveal electrode cluster differences between the
  ID and NRS groups and KDT conditions.
\item
  Clusters associated with higher SR during resting wake will be
  associated with lower delta power during sleep
\end{enumerate}

\subsection{Present study}\label{present-study}

This study will use HD-EEG to examine brain activity during the KDT to
examine the correlation with self-reported sleepiness. Hypotheses:

\subsection{Summary of chapters}\label{summary-of-chapters}

\newpage

\section{Method}\label{sec:method}

\subsection{Participants}\label{participants}

12 participants from each clinical population were recruited:
individuals with insomnia disorder (ID), individuals with
non-restorative sleep (NRS), and healthy controls (HC). Recruitment was
conducted through referrals from the Woolcock Institute and the Royal
Prince Alfred sleep clinics, in addition to social media advertising.
Due to the influence of age and sex on sleep architecture (Mongrain et
al., 2005), participants were sex and age matched with a maximum
difference of 1 year.

Participants were excluded if they had any comorbid sleep apnoea, as
measured by wrist oximetry (oxygen desaturation index above 10 during
any night of monitoring) (WristOX has high sensitivity of diagnosing
OSAS (Nigro et al., 2009)). Participants were additionally excluded if
they had clinically significant depression or anxiety scores as measured
through the DASS-21, heavy alcohol use, pregnancy, circadian rhythm
disruption through shift work or recent international travel, or a
natural sleep time that of less than 6 hours or outside the hours of
21:30 and 8:00. As medications are known to affect sleep architecture,
participants taking regular medications affecting sleep were excluded.

The inclusion criteria for the ID group was as set by the Diagnostic and
Statistical Manual of Mental Disorders, Fifth Edition (Association,
2022b) criteria, with difficulty initiating or maintaining sleep
persisting for over 1 month causing clinically significant distress or
impairment in daily life. They additionally were required to have a
Pittsburgh Sleep Quality Index (PSIQ) score of 6 or higher, and an
Insomnia Severity Index (ISI) score of 16 or higher.

Individuals in the NRS group could not have a mean Total Sleep Time
(TST) below six hours as measured by sleep diary or actigraphy, or a
mean refreshed score above 3. Inclusion in this group required a PSQI of
6 or more, with subcomponent scores of at least 2 on the PSQI Component
1 and 10 on PSQI Component 5.

Healthy controls needed to have a PSQI score of 4 or less and an ISI
score of 6 or less.

\textbf{consent} All patients were remunerated \$100 upon successful
completion of the study.

\subsection{Protocol}\label{protocol}

The study was approved by the Macquarie University Human Research Ethics
Committee. Participants attended the sleep laboratory at the Woolcock
Institute of Medical Research for inital screening by a sleep physician.
Participants baseline sleep and activity patterns were measured via an
Actigraphy watch (\textbf{which one}) for 7 days prior, which was
validated against self-reported sleep diaries. Participants additionally
completed the Restorative Sleep Questionnaire Daily Version (RSQ-D) for
7 days prior.

Upon arrival at the laboratory at 17:00, participants underwent final
medical screening and a series of cognitive assessment. They were then
served dinner and fitted with a high-density electroencephalography
(HD-EEG) cap. Further cognitive assessments were conducted before the
administration of the Karolinska Drowsiness Test (KDT) approximately 45
minutes prior to their habitual bedtime. Overnight polysomnography using
HD-EEG was recorded, in addition to sleep video recording using a AXIS
P3225-LV camera.

Lights were turned on at the participant's natural wake time and they
were asked if they were already awake or wakened by researchers. The KDT
was repeated five minutes post-habitual wake time. Following the morning
KDT, participants completed further cognitive testing and an MRI scan.

\subsection{Measures}\label{measures}

\subsubsection{KSS}\label{kss}

Subjective sleepiness was assessed using the Karolinska Sleepiness Scale
(KSS), a 9 point scale with verbal anchors at steps 1(``extremely
alert'')., 3 (``alert''), 5 (``neither alert nor sleepy''), 7
(``sleepy-but no difficulty remaining awake''), and 9 (Extremely
sleepy-fighting sleep) (Åkerstedt \& Gillberg, 1990) \textbf{confirm if
this kss or one with anchor at each step}. The KSS has been vaidated in
healthy populations as being closely related to EEG and behavioural
variables of sleepiness (Kaida et al., 2006).

\begin{itemize}
\tightlist
\item
  sensitive to manipulations known to affect sleepiness, correlate with
  impaired waking function and appear to be used consistently across
  individuals (Åkerstedt et al., 2014)
\end{itemize}

\subsubsection{KDT}\label{kdt}

The KDT was used to measure electrophysiological drowsiness as measured
through HD-EEG recordings. Participants were instructed ``Look at the
dot in front of you and be as relaxed as possible while staying awake.
Keep your head and body still and minimize blinking. After a few
minutes, I'll ask you to close your eyes and keep them closed for a few
minutes. Finally, I'll ask you to open your eyes again and keep them
open for a few minutes.'' They commence with their eyes open, close
their eyes at 2m10, open eyes again at 4m40, and the test ends at 7m10.

\subsubsection{HD EEG}\label{hd-eeg}

High-density EEG data were collected using 256-channel caps
(\textbf{which one}). Th
\texttt{signals\ were\ amplified\ and\ digitised,\ impedences,\ recordings\ were\ acquired\ with\ electrodes\ referenced\ to\ the\ vertex}
\texttt{processing\ of\ original\ eeg\ signals\ was\ performed}

The data was visually inspected for artefacts and arousals using a
\textbf{semi-automatic process} and was manually verified and cleaned.
\texttt{The\ record\ was\ visually\ inspected\ for\ bad\ channels\ and\ channels\ identified\ as\ poor\ quality\ (2.5\%\ ±\ 1.4\%\ of\ 164\ chan-\ nels)\ were\ replaced\ with\ an\ interpolated\ EEG\ signal\ using\ a\ spher-\ ical\ spline\ interpolation\ algorithm.\ After\ artifact\ removal\ and\ bad\ channel\ interpolation,\ the\ EEG\ signals\ were\ average-referenced.}
\textbf{did we do this?}

To calculate power spectral density, cleaned EEG signals were analysed
using a fast Fourier transform
\texttt{with\ 50\%\ overlapping\ between\ con-\ secutive\ 4-second\ windows\ with\ a\ Hanning\ filter\ function,\ resulting\ in\ a\ frequency\ resolution\ of\ 0.25\ Hz}

EEG spectral power densities were quantified as: delta (1--4.5 Hz),
theta (4.5--8 Hz), alpha (8--12 Hz), sigma (12--15 Hz), beta (15--25
Hz), and gamma (25--40 Hz).

\subsection{Statistical analysis}\label{statistical-analysis}

z-score normalised power spectral data were analysed for eyes open and
eyes closed conditions for each participant

\texttt{To\ control\ for\ Type\ I\ error\ rate\ in\ cluster\ analysis,\ statistical\ nonparametric\ mapping\ (SnPM)\ with\ the\ suprathreshold\ cluster\ test\ will\ be\ used.\ SnPM\ uses\ permutation\ tests\ (10\ 000\ random\ shuffles\ of\ the\ data)\ to\ establish\ a\ distribution\ of\ cluster\ size\ findings\ that\ occur\ due\ to\ chance.\ This\ distribution\ can\ then\ be\ used\ to\ compare\ cluster\ size\ to\ the\ a\ priori\ set\ threshold\ of\ p\ \textless{}\ .05,\ determining\ if\ it\ is\ statistically\ significant\ (D’Rozario\ et\ al.,\ 2023).}

EEG processor

All analyses were performed using MATLAB version r2024a (MathWorks,
Natick, MA, USA).

\newpage

\section{Results}\label{sec:results}

\subsection{Comparing KSS scores between
groups}\label{comparing-kss-scores-between-groups}

All analyses were conducted on Matlab version R2024a and EEGprocessor
\emph{version}.

\subsection{Correlation between KSS and slowing ratio scores between
groups}\label{correlation-between-kss-and-slowing-ratio-scores-between-groups}

\subsection{Correlation between KSS and AAC between
groups}\label{correlation-between-kss-and-aac-between-groups}

\subsection{Topography of channel-by-channel comparisons between ID and
NRS
groups}\label{topography-of-channel-by-channel-comparisons-between-id-and-nrs-groups}

\newpage

\section{Discussion}\label{sec:discussion}

The study aimed to explore the relationship between self-reported
sleepiness scores, as measured by the KSS, and neural markers of
drowsiness measured in the KDT across a sample of people with insomnia,
non-restorative sleep, and healthy controls.

\subsubsection{KSS score variance}\label{kss-score-variance}

The study found that KSS scores varied across groups.

\subsubsection{AAC}\label{aac}

This is how AAC scores correlated amongst 3 groups

\subsubsection{Slowing Ratio}\label{slowing-ratio}

Here I will talk about slowing ratio

\subsubsection{Topographic electrode cluster differences between
ID/NRS}\label{topographic-electrode-cluster-differences-between-idnrs}

Topographic power spectral analysis found these cluster differences
which mean this

\subsection{Strengths}\label{strengths}

\begin{itemize}
\tightlist
\item
  Age and sex matching of participants
\item
  Strong exclusion criteria
\end{itemize}

\subsection{Limitations}\label{limitations}

\begin{itemize}
\tightlist
\item
  Sample size
\end{itemize}

\subsection{Practical implications and future
directions}\label{practical-implications-and-future-directions}

\subsection{Conclusion}\label{conclusion}

The KSS is the best measure ever and more people should use it.

\newpage

\footnotesize
\singlespacing
\setlength{\parindent}{0in}

\section{References}\label{references}

\newpage

\section*{Appendix 1: Some extra
stuff}\label{appendix-1-some-extra-stuff}
\addcontentsline{toc}{section}{Appendix 1: Some extra stuff}

Add appendix 1 here. Vivamus hendrerit rhoncus interdum. Sed ullamcorper
et augue at porta. Suspendisse facilisis imperdiet urna, eu pellentesque
purus suscipit in. Integer dignissim mattis ex aliquam blandit.
Curabitur lobortis quam varius turpis ultrices egestas.

\newpage

\phantomsection\label{refs}
\begin{CSLReferences}{1}{0}
\bibitem[\citeproctext]{ref-akerstedt2014}
Åkerstedt, T., Anund, A., Axelsson, J., \& Kecklund, G. (2014).
Subjective sleepiness is a sensitive indicator of insufficient sleep and
impaired waking function. \emph{Journal of Sleep Research},
\emph{23}(3), 242--254. \url{https://doi.org/10.1111/jsr.12158}

\bibitem[\citeproctext]{ref-akerstedt1990}
Åkerstedt, T., \& Gillberg, M. (1990). Subjective and {Objective
Sleepiness} in the {Active Individual}. \emph{International Journal of
Neuroscience}, \emph{52}(1-2), 29--37.
\url{https://doi.org/10.3109/00207459008994241}

\bibitem[\citeproctext]{ref-akerstedt2009}
Åkerstedt, T., Kecklund, G., Ingre, M., Lekander, M., \& Axelsson, J.
(2009). Sleep {Homeostasis During Repeated Sleep Restriction} and
{Recovery}: {Support} from {EEG Dynamics}. \emph{Sleep}, \emph{32}(2),
217--222. \url{https://doi.org/10.5665/sleep/32.2.217}

\bibitem[\citeproctext]{ref-americanpsychiatricassociation2022}
Association, A. P. (2022a). \emph{Diagnostic and statistical manual of
mental disorders} (5th ed., text revision). American Psychiatric
Association.

\bibitem[\citeproctext]{ref-apa2022}
Association, A. P. (2022b). \emph{Diagnostic and statistical manual of
mental disorders} (5th ed., text revision). American Psychiatric
Association.

\bibitem[\citeproctext]{ref-bernardi2019}
Bernardi, G., Betta, M., Ricciardi, E., Pietrini, P., Tononi, G., \&
Siclari, F. (2019). Regional {Delta Waves In Human Rapid Eye Movement
Sleep}. \emph{Journal of Neuroscience}, \emph{39}(14), 2686--2697.
\url{https://doi.org/10.1523/JNEUROSCI.2298-18.2019}

\bibitem[\citeproctext]{ref-cantero2002}
Cantero, J. L., Atienza, M., \& Salas, R. M. (2002). Human alpha
oscillations in wakefulness, drowsiness period, and {REM} sleep:
Different electroencephalographic phenomena within the alpha band.
\emph{Neurophysiologie Clinique/Clinical Neurophysiology}, \emph{32}(1),
54--71. \url{https://doi.org/10.1016/S0987-7053(01)00289-1}

\bibitem[\citeproctext]{ref-cox2020}
Cox, R., \& Fell, J. (2020). Analyzing human sleep {EEG}: {A}
methodological primer with code implementation. \emph{Sleep Medicine
Reviews}, \emph{54}, 101353.
\url{https://doi.org/10.1016/j.smrv.2020.101353}

\bibitem[\citeproctext]{ref-hossain2005}
Hossain, J. L., Ahmad, P., Reinish, L. W., Kayumov, L., Hossain, N. K.,
\& Shapiro, C. M. (2005). Subjective fatigue and subjective sleepiness:
Two independent consequences of sleep disorders? \emph{Journal of Sleep
Research}, \emph{14}(3), 245--253.
\url{https://doi.org/10.1111/j.1365-2869.2005.00466.x}

\bibitem[\citeproctext]{ref-jap2009}
Jap, B. T., Lal, S., Fischer, P., \& Bekiaris, E. (2009). Using {EEG}
spectral components to assess algorithms for detecting fatigue.
\emph{Expert Systems with Applications}, \emph{36}(2, Part 1),
2352--2359. \url{https://doi.org/10.1016/j.eswa.2007.12.043}

\bibitem[\citeproctext]{ref-kaida2006}
Kaida, K., Takahashi, M., Akerstedt, T., Nakata, A., Otsuka, Y.,
Haratani, T., \& Fukasawa, K. (2006). Validation of the {Karolinska}
sleepiness scale against performance and {EEG} variables. \emph{Clinical
Neurophysiology: Official Journal of the International Federation of
Clinical Neurophysiology}, \emph{117}(7), 1574--1581.
\url{https://doi.org/10.1016/j.clinph.2006.03.011}

\bibitem[\citeproctext]{ref-kao2021}
Kao, C.-H., D'Rozario, A. L., Lovato, N., Wassing, R., Bartlett, D.,
Memarian, N., Espinel, P., Kim, J.-W., Grunstein, R. R., \& Gordon, C.
J. (2021). Insomnia subtypes characterised by objective sleep duration
and {NREM} spectral power and the effect of acute sleep restriction: An
exploratory analysis. \emph{Scientific Reports}, \emph{11}(1), 24331.
\url{https://doi.org/10.1038/s41598-021-03564-6}

\bibitem[\citeproctext]{ref-li2016}
Li, Y., Liu, H., Weed, J. G., Ren, R., Sun, Y., Tan, L., \& Tang, X.
(2016). Deficits in attention performance are associated with
insufficiency of slow-wave sleep in insomnia. \emph{Sleep Medicine},
\emph{24}, 124--130. \url{https://doi.org/10.1016/j.sleep.2016.07.017}

\bibitem[\citeproctext]{ref-lunsford-avery2021}
Lunsford-Avery, J. R., Edinger, J. D., \& Krystal, A. D. (2021).
Optimizing computation of overnight decline in delta power: {Evidence}
for slower rate of decline in delta power in insomnia patients.
\emph{Clinical Neurophysiology}, \emph{132}(2), 545--553.
\url{https://doi.org/10.1016/j.clinph.2020.12.004}

\bibitem[\citeproctext]{ref-massimini2004}
Massimini, M., Huber, R., Ferrarelli, F., Hill, S., \& Tononi, G.
(2004). The {Sleep Slow Oscillation} as a {Traveling Wave}.
\emph{Journal of Neuroscience}, \emph{24}(31), 6862--6870.
\url{https://doi.org/10.1523/JNEUROSCI.1318-04.2004}

\bibitem[\citeproctext]{ref-mongrain2005}
Mongrain, V., Carrier, J., \& Dumont, M. (2005). Chronotype and {Sex
Effects} on {Sleep Architecture} and {Quantitative Sleep EEG} in
{Healthy Young Adults}. \emph{Sleep}, \emph{28}(7), 819--827.
\url{https://doi.org/10.1093/sleep/28.7.819}

\bibitem[\citeproctext]{ref-nigro2009}
Nigro, C. A., Aimaretti, S., Gonzalez, S., \& Rhodius, E. (2009).
Validation of the {WristOx} 3100™ oximeter for the diagnosis of sleep
apnea/hypopnea syndrome. \emph{Sleep and Breathing}, \emph{13}(2),
127--136. \url{https://doi.org/10.1007/s11325-008-0217-3}

\bibitem[\citeproctext]{ref-reynolds2019}
Reynolds, A., Appleton, S. L., Gill, T. K., \& Adams, R. (2019).
\emph{Chronic insomnia disorder in {Australia}: {A} report to the {Sleep
Health Foundation}}. Flux Visual Communication.

\bibitem[\citeproctext]{ref-roth2010}
Roth, T., Zammit, G., Lankford, A., Mayleben, D., Stern, T., Pitman, V.,
Clark, D., \& Werth, J. L. (2010). Nonrestorative {Sleep} as a {Distinct
Component} of {Insomnia}. \emph{Sleep}, \emph{33}(4), 449--458.
\url{https://doi.org/10.1093/sleep/33.4.449}

\bibitem[\citeproctext]{ref-steriade1993}
Steriade, M., Nunez, A., \& Amzica, F. (1993). A novel slow (\&lt; 1
{Hz}) oscillation of neocortical neurons in vivo: Depolarizing and
hyperpolarizing components. \emph{Journal of Neuroscience},
\emph{13}(8), 3252--3265.
\url{https://doi.org/10.1523/JNEUROSCI.13-08-03252.1993}

\bibitem[\citeproctext]{ref-steriade2001}
Steriade, M., Timofeev, I., \& Grenier, F. (2001). Natural {Waking} and
{Sleep States}: {A View From Inside Neocortical Neurons}. \emph{Journal
of Neurophysiology}, \emph{85}(5), 1969--1985.
\url{https://doi.org/10.1152/jn.2001.85.5.1969}

\bibitem[\citeproctext]{ref-stone2008}
Stone, K. C., Taylor, D. J., McCrae, C. S., Kalsekar, A., \& Lichstein,
K. L. (2008). Nonrestorative sleep. \emph{Sleep Medicine Reviews},
\emph{12}(4), 275--288. \url{https://doi.org/10.1016/j.smrv.2007.12.002}

\bibitem[\citeproctext]{ref-sweetman2021}
Sweetman, A., Melaku, Y. A., Lack, L., Reynolds, A., Gill, T. K., Adams,
R., \& Appleton, S. (2021). Prevalence and associations of co-morbid
insomnia and sleep apnoea in an {Australian} population-based sample.
\emph{Sleep Medicine}, \emph{82}, 9--17.
\url{https://doi.org/10.1016/j.sleep.2021.03.023}

\bibitem[\citeproctext]{ref-wu2013}
Wu, Y. M., Pietrone, R., Cashmere, J. D., Begley, A., Miewald, J. M.,
Germain, A., \& Buysse, D. J. (2013). {EEG Power During Waking} and
{NREM Sleep} in {Primary Insomnia}. \emph{Journal of Clinical Sleep
Medicine}, \emph{09}(10), 1031--1037.
\url{https://doi.org/10.5664/jcsm.3076}

\bibitem[\citeproctext]{ref-zhao2021}
Zhao, W., Van Someren, E. J. W., Li, C., Chen, X., Gui, W., Tian, Y.,
Liu, Y., \& Lei, X. (2021). {EEG} spectral analysis in insomnia
disorder: {A} systematic review and meta-analysis. \emph{Sleep Medicine
Reviews}, \emph{59}, 101457.
\url{https://doi.org/10.1016/j.smrv.2021.101457}

\end{CSLReferences}

\end{document}
