\documentclass[12pt,a4paper,]{report}
\usepackage{lmodern}

% Overwrite \begin{figure}[htbp] with \begin{figure}[H]
\usepackage{float}
\let\origfigure=\figure
\let\endorigfigure=\endfigure
\renewenvironment{figure}[1][]{%
\origfigure[b]
}{%
\endorigfigure
}

% fix for pandoc 1.14
\providecommand{\tightlist}{%
  \setlength{\itemsep}{0pt}\setlength{\parskip}{0pt}}

% TP: hack to truncate list of figures/tables.
\usepackage{truncate}
\usepackage{caption}
\usepackage{tocloft}
% TP: end hack

\usepackage{amssymb,amsmath}
\usepackage{ifxetex,ifluatex}

% Only use fixltx2e if using pre-2015 kernels
\begingroup\expandafter\expandafter\expandafter\endgroup
\expandafter\ifx\csname IncludeInRelease\endcsname\relax
  \usepackage{fixltx2e}
\fi

\ifnum 0\ifxetex 1\fi\ifluatex 1\fi=0 % if pdftex
  \usepackage[T1]{fontenc}
  \usepackage[utf8]{inputenc}
\mathbf{
\else % if luatex or xelatex
  \ifxetex
    \usepackage{mathspec}
    \usepackage{xltxtra,xunicode}
  \else
    \usepackage{fontspec}
  \fi
  \defaultfontfeatures{Mapping=tex-text,Scale=MatchLowercase}
  \newcommand{\euro}{€}
\fi
% use upquote if available, for straight quotes in verbatim environments
\IfFileExists{upquote.sty}{\usepackage{upquote}}{}
% use microtype if available
\IfFileExists{microtype.sty}{%
\usepackage{microtype}
\UseMicrotypeSet[protrusion]{basicmath} % disable protrusion for tt fonts
}{}
\usepackage{longtable,booktabs}
\ifxetex
  \usepackage[setpagesize=false, % page size defined by xetex
              unicode=false, % unicode breaks when used with xetex
              xetex]{hyperref}
\else
  \usepackage[unicode=true]{hyperref}
\fi
\hypersetup{breaklinks=true,
            bookmarks=true,
            pdfauthor={Anastasia Stuart},
            pdftitle={How sleepy are you?},
            colorlinks=true,
            citecolor=blue,
            urlcolor=blue,
            linkcolor=magenta,
            pdfborder={0 0 0}}
\urlstyle{same}  % don't use monospace font for urls
\setlength{\parindent}{0pt}
\setlength{\parskip}{6pt plus 2pt minus 1pt}
\setlength{\emergencystretch}{3em}  % prevent overfull lines
\setcounter{secnumdepth}{5}

% % \newlength{\cslhangindent}
% \setlength{\cslhangindent}{1.5em}
% \newenvironment{cslreferences}%
%   {}%
%   {\par}
% 
\newlength{\cslhangindent}
\setlength{\cslhangindent}{1.5em}
\newlength{\csllabelwidth}
\setlength{\csllabelwidth}{3em}
\newenvironment{CSLReferences}[2] % #1 hanging-ident, #2 entry spacing
 {% don't indent paragraphs
  \setlength{\parindent}{0pt}
  % turn on hanging indent if param 1 is 1
  \ifodd #1 \everypar{\setlength{\hangindent}{\cslhangindent}}\ignorespaces\fi
  % set entry spacing
  \ifnum #2 > 0
  \setlength{\parskip}{#2\baselineskip}
  \fi
 }%
 {}
\usepackage{calc}
\newcommand{\CSLBlock}[1]{#1\hfill\break}
\newcommand{\CSLLeftMargin}[1]{\parbox[t]{\csllabelwidth}{#1}}
\newcommand{\CSLRightInline}[1]{\parbox[t]{\linewidth - \csllabelwidth}{#1}\break}
\newcommand{\CSLIndent}[1]{\hspace{\cslhangindent}#1}



% Table of contents formatting
\renewcommand{\contentsname}{Table of Contents}
\setcounter{tocdepth}{3}

% Headers and page numbering
\usepackage{fancyhdr}
\pagestyle{plain}

% Following package is used to add background image to front page
\usepackage{wallpaper}

% Table package
\usepackage{ctable}% http://ctan.org/pkg/ctable

% Deal with 'LaTeX Error: Too many unprocessed floats.'
\usepackage{morefloats}
% or use \extrafloats{100}
% add some \clearpage

% % Chapter header
% \usepackage{titlesec, blindtext, color}
% \definecolor{gray75}{gray}{0.75}
% \newcommand{\hsp}{\hspace{20pt}}
% \titleformat{\chapter}[hang]{\Huge\bfseries}{\thechapter\hsp\textcolor{gray75}{|}\hsp}{0pt}{\Huge\bfseries}

% % Fonts and typesetting
% \setmainfont[Scale=1.1]{Helvetica}
% \setsansfont[Scale=1.1]{Verdana}

% FONTS
\usepackage{xunicode}
\usepackage{xltxtra}
\defaultfontfeatures{Mapping=tex-text} % converts LaTeX specials (``quotes'' --- dashes etc.) to unicode
% \setromanfont[Scale=1.01,Ligatures={Common},Numbers={OldStyle}]{Palatino}
% \setromanfont[Scale=1.01,Ligatures={Common},Numbers={OldStyle}]{Adobe Caslon Pro}
%Following line controls size of code chunks
% \setmonofont[Scale=0.9]{Monaco}
%Following line controls size of figure legends
% \setsansfont[Scale=1.2]{Optima Regular}

% CODE BLOCKS
\usepackage[utf8]{inputenc}
\usepackage{listings}
\usepackage{color}

% JAVA CODE BLOCKS
%\definecolor{backcolour}{RGB}{242,242,242}
%\definecolor{javared}{rgb}{0.6,0,0}
%\definecolor{javagreen}{rgb}{0.25,0.5,0.35}
%\definecolor{javapurple}{rgb}{0.5,0,0.35}
%\definecolor{javadocblue}{rgb}{0.25,0.35,0.75}

\lstdefinestyle{javaCodeStyle}{
  language=Java,                         % the language of the code
  backgroundcolor=\color{backcolour},    % choose the background color; you must add \usepackage{color} or \usepackage{xcolor}
  basicstyle=\fontsize{10}{8}\sffamily,
  breakatwhitespace=false,
  breaklines=true,
  keywordstyle=\color{javapurple}\bfseries,
  stringstyle=\color{javared},
  commentstyle=\color{javagreen},
  morecomment=[s][\color{javadocblue}]{/**}{*/},
  captionpos=t,                          % sets the caption-position to bottom
  frame=single,                          % adds a frame around the code
  numbers=left,
  numbersep=10pt,                         % margin between number and code block
  keepspaces=true,                       % keeps spaces in text, useful for keeping indentation of code (possibly needs columns=flexible)
  columns=fullflexible,
  showspaces=false,                      % show spaces everywhere adding particular underscores; it overrides 'showstringspaces'
  showstringspaces=false,                % underline spaces within strings only
  showtabs=false,                        % show tabs within strings adding particular underscores
  tabsize=2                              % sets default tabsize to 2 spaces
}

%Attempt to set math size
%First size must match the text size in the document or command will not work
%\DeclareMathSizes{display size}{text size}{script size}{scriptscript size}.
%\DeclareMathSizes{12}{13}{7}{7}

% ---- CUSTOM AMPERSAND
% \newcommand{\amper}{{\fontspec[Scale=.95]{Adobe Caslon Pro}\selectfont\itshape\&}}

% HEADINGS
\usepackage{sectsty}
\usepackage[normalem]{ulem}
\sectionfont{\rmfamily\mdseries\Large}
\subsectionfont{\rmfamily\mdseries\scshape\large}
\subsubsectionfont{\rmfamily\bfseries\upshape\large}
% \sectionfont{\rmfamily\mdseries\Large}
% \subsectionfont{\rmfamily\mdseries\scshape\normalsize}
% \subsubsectionfont{\rmfamily\bfseries\upshape\normalsize}

% Set figure legends and captions to be smaller sized sans serif font
\usepackage[font={footnotesize,sf}]{caption}

\usepackage{siunitx}

% Adjust spacing between lines to 1.5
\usepackage{setspace}
% \onehalfspacing
\doublespacing
\raggedbottom

% Set margins
\usepackage[top=1.5in,bottom=1.5in,left=1.5in,right=1.4in]{geometry}
\setlength\parindent{0.5in} % indent at start of paragraphs (set to 0.3?)
\setlength{\parskip}{9pt}
\usepackage{indentfirst}

% Add space between pararaphs
% http://texblog.org/2012/11/07/correctly-typesetting-paragraphs-in-latex/
% \usepackage{parskip}
% \setlength{\parskip}{\baselineskip}

% Set colour of links to black so that they don't show up when printed
\usepackage{hyperref}
\hypersetup{colorlinks=false, linkcolor=black}

% Tables
\usepackage{booktabs}
\usepackage{threeparttable}
\usepackage{array}
\usepackage{makecell}
\newcolumntype{x}[1]{%
>{\centering\arraybackslash}m{#1}}%

% Allow for long captions and float captions on opposite page of figures
% \usepackage[rightFloats, CaptionBefore]{fltpage}

% Don't let floats cross subsections
% \usepackage[section,subsection]{extraplaceins}

% Rotate images and tables
\usepackage{float}
\usepackage{pdfpages}
\usepackage{pdflscape}
\usepackage{graphicx}
\usepackage{rotating}

% Custom math
\usepackage{bbold}
\DeclareMathOperator*{\argmin}{\arg\!\min}

% pandoc-crossref definitions

% We add the lines below from pandoc-crossref because we are using the --include-in-header flag
% see here: https://lierdakil.github.io/pandoc-crossref/#latex-output-and---include-in-header
% This LaTeX code is obtained by getting pandoc-crossref to dump it
% see here: https://github.com/lierdakil/pandoc-crossref/issues/326


\begin{document}


\begin{titlepage}
    \begin{center}


        \vspace*{2.5cm}

        \huge
        How sleepy are you?

                \vspace{.5cm}

        \Large
        KSS and KDT in Insomnia and Non-Restorative Sleep
        

        \vspace{1.5cm}

        \Large
        Anastasia Stuart

        \vspace{1.5cm}

        \normalsize
        A thesis presented for the degree of\\
        Bachelor of Psychology (Honours) 2024

        \vfill

        \normalsize
        Supervised by:\\
        Dr Rick Wassing \\ Dr Julia Chapman

        \vspace{0.8cm}

        % Uncomment the following line
        % to add a centered university logo
        % \includegraphics[width=0.4\textwidth]{style/univ_logo.eps}

        \normalsize
        Macquarie University\\
        October 2024

        % Except where otherwise noted, content in this thesis is licensed under a Creative Commons Attribution 4.0 License (http://creativecommons.org/licenses/by/4.0), which permits unrestricted use, distribution, and reproduction in any medium, provided the original work is properly cited. Copyright 2015,Tom Pollard.

    \end{center}
\end{titlepage}


% This is where the converted markdown files will go 
\vspace*{\fill}

\noindent \textit{
I, Anastasia Stuart confirm that the work presented in this thesis is my own. Where information has been derived from other sources, I confirm that this has been indicated in the thesis.
} \vspace*{\fill} \pagenumbering{gobble} \newpage

\chapter*{Acknowledgements}\label{acknowledgements}
\addcontentsline{toc}{chapter}{Acknowledgements}

\newpage

\chapter*{Abstract}\label{abstract}
\addcontentsline{toc}{chapter}{Abstract}

Lorem ipsum dolor sit amet, consectetur adipiscing elit. Nam et turpis
gravida, lacinia ante sit amet, sollicitudin erat. Aliquam efficitur
vehicula leo sed condimentum. Phasellus lobortis eros vitae rutrum
egestas. Vestibulum ante ipsum primis in faucibus orci luctus et
ultrices posuere cubilia Curae; Donec at urna imperdiet, vulputate orci
eu, sollicitudin leo. Donec nec dui sagittis, malesuada erat eget,
vulputate tellus. Nam ullamcorper efficitur iaculis. Mauris eu vehicula
nibh. In lectus turpis, tempor at felis a, egestas fermentum massa.

\newpage 
\pagenumbering{roman}
\setcounter{page}{1}

\pagenumbering{gobble}

\tableofcontents

\newpage

\listoffigures

\newpage

\listoftables

\newpage

\chapter*{Abbreviations}\label{abbreviations}
\addcontentsline{toc}{chapter}{Abbreviations}

\begin{tabbing}
\textbf{API}~~~~~~~~~~~~ \= \textbf{A}pplication \textbf{P}rogramming \textbf{I}nterface \\  
\textbf{JSON} \> \textbf{J}ava\textbf{S}cript \textbf{O}bject \textbf{N}otation \\  
\end{tabbing}

\newpage

\setcounter{page}{1}
\pagenumbering{arabic}
\doublespacing
\setlength{\parindent}{0.5in}

\chapter{Introduction}\label{introduction}

\section{Background}\label{background}

Recent research suggests there may be a distinct subtype of insomnia
called non-restorative sleep, characterized by sleep-state
misperception.

\section{Sleep-state misperception}\label{sleep-state-misperception}

\subsection{Subsection of the middle
bit}\label{subsection-of-the-middle-bit}

This is a subsection of the middle bit. Quisque sit amet tempus arcu, ac
suscipit ante. Cras massa elit, pellentesque eget nisl ut, malesuada
rutrum risus. Nunc in venenatis mi. Curabitur sit amet suscipit eros,
non tincidunt nibh. Phasellus lorem lectus, iaculis non luctus eget,
tempus non risus. Suspendisse ut felis mi. (Sweetman et al., 2021)

\section{Self-reported sleepiness}\label{self-reported-sleepiness}

Self-reported sleepiness can be measured by the Karolinska sleepiness
scale, which correlates to neural measures of drowsiness in healthy
controls (Kaida et al., 2006)

\subsection{Subheading}\label{subheading}

\section{Summary of chapters}\label{summary-of-chapters}

\newpage

\chapter{Method}\label{sec:method}

\section{Participants}\label{participants}

12 participants from each clinical group (ID, NRS, HC) were recruited
through referrals from the Woolcock Institute and the Royal Prince
Alfred sleep clinics in addition to social media advertising. - Age and
sex matched - excluded if comorbid sleep disorder - Inclusion criteria
for insomnia - inclusion criteria for NRS - Remunerated \$100

\section{Measures}\label{measures}

\subsection{KSS}\label{kss}

\begin{itemize}
\tightlist
\item
  KSS is a 1 item 9-point likert scale measure
\item
  internal and external validity
\item
  measures sleepiness
\end{itemize}

\subsection{KDT}\label{kdt}

\begin{itemize}
\tightlist
\item
  KDT measured through HD-EEG data
\item
  Eyes open and eyes closed conditions
\item
  Power spectra
\end{itemize}

\section{Procedure}\label{procedure}

The study was approved by the Macquarie University Human Research Ethics
Committee. - Participants come to the Woolcock - Sleep is monitored
overnight - KSS and KDT recorded at 7am and 9am - Other neurobehavioural
testing also done

\newpage

\chapter{Results}\label{sec:results}

\section{Comparing KSS scores between
groups}\label{comparing-kss-scores-between-groups}

All analyses were conducted on Matlab version R2024a and EEGprocessor
\emph{version}.

\section{Correlation between KSS and slowing ratio scores between
groups}\label{correlation-between-kss-and-slowing-ratio-scores-between-groups}

\section{Correlation between KSS and AAC between
groups}\label{correlation-between-kss-and-aac-between-groups}

\section{Topography of channel-by-channel comparisons between ID and NRS
groups}\label{topography-of-channel-by-channel-comparisons-between-id-and-nrs-groups}

\newpage

\chapter{Discussion}\label{sec:discussion}

The study aimed to explore the relationship between self-reported
sleepiness scores, as measured by the KSS, and neural markers of
drowsiness measured in the KDT across a sample of people with insomnia,
non-restorative sleep, and healthy controls.

\subsection{KSS score variance}\label{kss-score-variance}

The study found that KSS scores varied across groups.

\subsection{AAC}\label{aac}

This is how AAC scores correlated amongst 3 groups

\subsection{Slowing Ratio}\label{slowing-ratio}

Here I will talk about slowing ratio

\subsection{Topographic electrode cluster differences between
ID/NRS}\label{topographic-electrode-cluster-differences-between-idnrs}

Topographic power spectral analysis found these cluster differences
which mean this

\section{Strengths}\label{strengths}

\begin{itemize}
\tightlist
\item
  Age and sex matching of participants
\item
  Strong exclusion criteria
\end{itemize}

\section{Limitations}\label{limitations}

\begin{itemize}
\tightlist
\item
  Sample size
\end{itemize}

\section{Practical implications and future
directions}\label{practical-implications-and-future-directions}

\section{Conclusion}\label{conclusion}

The KSS is the best measure ever and more people should use it.

\newpage

\footnotesize
\singlespacing
\setlength{\parindent}{0in}

\chapter{References}\label{references}

\newpage

\chapter*{Appendix 1: Some extra
stuff}\label{appendix-1-some-extra-stuff}
\addcontentsline{toc}{chapter}{Appendix 1: Some extra stuff}

Add appendix 1 here. Vivamus hendrerit rhoncus interdum. Sed ullamcorper
et augue at porta. Suspendisse facilisis imperdiet urna, eu pellentesque
purus suscipit in. Integer dignissim mattis ex aliquam blandit.
Curabitur lobortis quam varius turpis ultrices egestas.

\newpage

\vspace*{\fill}

\noindent \textit{
I, Anastasia Stuart confirm that the work presented in this thesis is my own. Where information has been derived from other sources, I confirm that this has been indicated in the thesis.
} \vspace*{\fill} \pagenumbering{gobble} \newpage

\chapter*{Acknowledgements}\label{acknowledgements-1}
\addcontentsline{toc}{chapter}{Acknowledgements}

\newpage

\chapter*{Abstract}\label{abstract-1}
\addcontentsline{toc}{chapter}{Abstract}

Lorem ipsum dolor sit amet, consectetur adipiscing elit. Nam et turpis
gravida, lacinia ante sit amet, sollicitudin erat. Aliquam efficitur
vehicula leo sed condimentum. Phasellus lobortis eros vitae rutrum
egestas. Vestibulum ante ipsum primis in faucibus orci luctus et
ultrices posuere cubilia Curae; Donec at urna imperdiet, vulputate orci
eu, sollicitudin leo. Donec nec dui sagittis, malesuada erat eget,
vulputate tellus. Nam ullamcorper efficitur iaculis. Mauris eu vehicula
nibh. In lectus turpis, tempor at felis a, egestas fermentum massa.

\newpage

\pagenumbering{gobble}

\tableofcontents

\newpage

\listoffigures

\newpage

\chapter*{Abbreviations}\label{abbreviations-1}
\addcontentsline{toc}{chapter}{Abbreviations}

\begin{tabbing}
\textbf{API}~~~~~~~~~~~~ \= \textbf{A}pplication \textbf{P}rogramming \textbf{I}nterface \\  
\textbf{JSON} \> \textbf{J}ava\textbf{S}cript \textbf{O}bject \textbf{N}otation \\  
\end{tabbing}

\newpage

\pagenumbering{roman}
\setcounter{page}{1}

\setcounter{page}{1}
\pagenumbering{arabic}
\doublespacing
\setlength{\parindent}{0.5in}

\chapter{Introduction}\label{introduction-1}

\section{Background}\label{background-1}

Recent research suggests there may be a distinct subtype of insomnia
called non-restorative sleep, characterized by sleep-state
misperception.

\section{Sleep-state misperception}\label{sleep-state-misperception-1}

\subsection{Subsection of the middle
bit}\label{subsection-of-the-middle-bit-1}

This is a subsection of the middle bit. Quisque sit amet tempus arcu, ac
suscipit ante. Cras massa elit, pellentesque eget nisl ut, malesuada
rutrum risus. Nunc in venenatis mi. Curabitur sit amet suscipit eros,
non tincidunt nibh. Phasellus lorem lectus, iaculis non luctus eget,
tempus non risus. Suspendisse ut felis mi. (Sweetman et al., 2021)

\section{Self-reported sleepiness}\label{self-reported-sleepiness-1}

Self-reported sleepiness can be measured by the Karolinska sleepiness
scale, which correlates to neural measures of drowsiness in healthy
controls (Kaida et al., 2006)

\subsection{Subheading}\label{subheading-1}

\section{Summary of chapters}\label{summary-of-chapters-1}

\newpage

\chapter{Method}\label{sec:method}

\section{Participants}\label{participants-1}

12 participants from each clinical group (ID, NRS, HC) were recruited
through referrals from the Woolcock Institute and the Royal Prince
Alfred sleep clinics in addition to social media advertising. - Age and
sex matched - excluded if comorbid sleep disorder - Inclusion criteria
for insomnia - inclusion criteria for NRS - Remunerated \$100

\section{Measures}\label{measures-1}

\subsection{KSS}\label{kss-1}

\begin{itemize}
\tightlist
\item
  KSS is a 1 item 9-point likert scale measure
\item
  internal and external validity
\item
  measures sleepiness
\end{itemize}

\subsection{KDT}\label{kdt-1}

\begin{itemize}
\tightlist
\item
  KDT measured through HD-EEG data
\item
  Eyes open and eyes closed conditions
\item
  Power spectra
\end{itemize}

\section{Procedure}\label{procedure-1}

The study was approved by the Macquarie University Human Research Ethics
Committee. - Participants come to the Woolcock - Sleep is monitored
overnight - KSS and KDT recorded at 7am and 9am - Other neurobehavioural
testing also done

\newpage

\chapter{Results}\label{sec:results}

\section{Comparing KSS scores between
groups}\label{comparing-kss-scores-between-groups-1}

All analyses were conducted on Matlab version R2024a and EEGprocessor
\emph{version}.

\section{Correlation between KSS and slowing ratio scores between
groups}\label{correlation-between-kss-and-slowing-ratio-scores-between-groups-1}

\section{Correlation between KSS and AAC between
groups}\label{correlation-between-kss-and-aac-between-groups-1}

\section{Topography of channel-by-channel comparisons between ID and NRS
groups}\label{topography-of-channel-by-channel-comparisons-between-id-and-nrs-groups-1}

\newpage

\chapter{Discussion}\label{sec:discussion}

The study aimed to explore the relationship between self-reported
sleepiness scores, as measured by the KSS, and neural markers of
drowsiness measured in the KDT across a sample of people with insomnia,
non-restorative sleep, and healthy controls.

\subsection{KSS score variance}\label{kss-score-variance-1}

The study found that KSS scores varied across groups.

\subsection{AAC}\label{aac-1}

This is how AAC scores correlated amongst 3 groups

\subsection{Slowing Ratio}\label{slowing-ratio-1}

Here I will talk about slowing ratio

\subsection{Topographic electrode cluster differences between
ID/NRS}\label{topographic-electrode-cluster-differences-between-idnrs-1}

Topographic power spectral analysis found these cluster differences
which mean this

\section{Strengths}\label{strengths-1}

\begin{itemize}
\tightlist
\item
  Age and sex matching of participants
\item
  Strong exclusion criteria
\end{itemize}

\section{Limitations}\label{limitations-1}

\begin{itemize}
\tightlist
\item
  Sample size
\end{itemize}

\section{Practical implications and future
directions}\label{practical-implications-and-future-directions-1}

\section{Conclusion}\label{conclusion-1}

The KSS is the best measure ever and more people should use it.

\newpage

\footnotesize
\singlespacing
\setlength{\parindent}{0in}

\chapter{References}\label{references-1}

\newpage

\chapter*{Appendix 1: Some extra
stuff}\label{appendix-1-some-extra-stuff-1}
\addcontentsline{toc}{chapter}{Appendix 1: Some extra stuff}

Add appendix 1 here. Vivamus hendrerit rhoncus interdum. Sed ullamcorper
et augue at porta. Suspendisse facilisis imperdiet urna, eu pellentesque
purus suscipit in. Integer dignissim mattis ex aliquam blandit.
Curabitur lobortis quam varius turpis ultrices egestas.

\newpage

\chapter*{Appendix 2: Some more extra
stuff}\label{appendix-2-some-more-extra-stuff}
\addcontentsline{toc}{chapter}{Appendix 2: Some more extra stuff}

Add appendix 2 here. Aliquam rhoncus mauris ac neque imperdiet, in
mattis eros aliquam. Etiam sed massa et risus posuere rutrum vel et
mauris. Integer id mauris sed arcu venenatis finibus. Etiam nec
hendrerit purus, sed cursus nunc. Pellentesque ac luctus magna. Aenean
non posuere enim, nec hendrerit lacus. Etiam lacinia facilisis tempor.
Aenean dictum nunc id felis rhoncus aliquam.

\newpage

\phantomsection\label{refs}
\begin{CSLReferences}{1}{0}
\bibitem[\citeproctext]{ref-kaida_validation_2006}
Kaida, K., Takahashi, M., Akerstedt, T., Nakata, A., Otsuka, Y.,
Haratani, T., \& Fukasawa, K. (2006). Validation of the {Karolinska}
sleepiness scale against performance and {EEG} variables. \emph{Clinical
Neurophysiology: Official Journal of the International Federation of
Clinical Neurophysiology}, \emph{117}(7), 1574--1581.
\url{https://doi.org/10.1016/j.clinph.2006.03.011}

\bibitem[\citeproctext]{ref-sweetman_prevalence_2021}
Sweetman, A., Melaku, Y. A., Lack, L., Reynolds, A., Gill, T. K., Adams,
R., \& Appleton, S. (2021). Prevalence and associations of co-morbid
insomnia and sleep apnoea in an {Australian} population-based sample.
\emph{Sleep Medicine}, \emph{82}, 9--17.
\url{https://doi.org/10.1016/j.sleep.2021.03.023}

\end{CSLReferences}

\end{document}
